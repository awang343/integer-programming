\documentclass[11pt]{article}

\usepackage[round]{natbib}
\usepackage{fullpage}
\usepackage[top=2cm, bottom=4.5cm, left=2.5cm, right=2.5cm]{geometry}
\usepackage{amsmath,amsthm,amsfonts,amssymb,amscd}
\usepackage{algorithm}
\usepackage{algpseudocode}
\usepackage{lastpage}
\usepackage{enumerate}
\usepackage{fancyhdr}
\usepackage{mathrsfs}
\usepackage{xcolor}
\usepackage{graphicx}
\usepackage{listings}
\usepackage{hyperref}

\newcommand\todo[1]{\textcolor{red}{#1}}

\hypersetup{%
  colorlinks=true,
  linkcolor=blue,
  linkbordercolor={0 0 1}
}

\renewcommand{\algorithmicrequire}{\textbf{Input:}}
\renewcommand{\algorithmicensure}{\textbf{Output:}}

\renewcommand\lstlistingname{Code}
\renewcommand\lstlistlistingname{Code}
\def\lstlistingautorefname{Code}

\lstdefinestyle{Python}{
    language        = Python,
    frame           = lines, 
    numbers=left,
    basicstyle      = \footnotesize,
    keywordstyle    = \color{blue},
    stringstyle     = \color{red},
    commentstyle    = \color{green}\ttfamily
}

\setlength{\parindent}{0.0in}
\setlength{\parskip}{0.0in}

%%%%%%%%%%%%% INPUT HERE
\newcommand\COURSE{CSCI2951-O}
\newcommand\PROJECTNUMBER{IV}
\newcommand\FULLNAMEONE{John Wu}
\newcommand\FULLNAMETWO{Alan Wang}
\newcommand\CSLOGINS{jwu175 \quad awang343}       
\newcommand\SCREENNAME{CookedPotato \quad null} 

\pagestyle{fancyplain}
\headheight 35pt
\lhead{\FULLNAMEONE\\\FULLNAMETWO}
\chead{\textbf{\Large Project -- \PROJECTNUMBER}}
\rhead{\CSLOGINS \\ \SCREENNAME}
\lfoot{}
\cfoot{}
\rfoot{\small\thepage}
\headsep 1.5em

%%%%%%%%%%%%% INPUT HERE
\begin{document}

As a team of two people, we spent approximately $24$ person-hours.
\vspace{-10pt}
\section*{Model Description}
\label{sec:intro}
\subsection*{Setup}
\begin{itemize}
    \item \textbf{Tests} - This was an array of binary variables representing whether or not
    we decide to test each symptom
    \item \textbf{XOR Matrix}: We created a 3d array of dimension \texttt{numDisease} by
    \texttt{numDisease} by \texttt{numTest}. Effectively, for each pair of diseases, we had a vector
    representing which tests could differentiate between those two diseases. This is precomputed.
\end{itemize}

\subsection*{Constraints and Objective}

\begin{itemize}
    \item \textbf{All Differentiable} - For each unique pair of diseases, we had to check that the dot
    product of which tests are activated with the XOR vector in the XOR Matrix was $\geq 1$. This
    guarantees that there is at least one active test that differentiates every pair of diseases.
    \item \textbf{Minimized Cost} - We took the dot product of the active tests vector and the
    cost vector to obtain the cost for a given assignment of tests. Our model aimed to minimize this.
\end{itemize}

\subsection*{Relaxation to LP}
We relaxed the binary variable for tests into a continuous variable from $0$ to $1$. This model 
produces a lower bound on the cost of testing, and we were able to use it at every node in our
search tree to get a cost lower bound given a set of forced assignments.

\section*{Branch and Bound Searching}
T

\end{document}
