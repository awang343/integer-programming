\subsection*{Metrics and Quality Assessment}

To assess the quality of our solutions we added in KPIs into our model to track different metrics in our schedule 
and then used these metrics to compare the quality of our solutions. We used the following metrics:
\begin{itemize}
    \item \textbf{Fairness: Total Hours Worked (per Employee):} The total number of hours worked over the schedule per employee. A example output looks like \texttt{[80, 80, 80, 80, 80, 80, 79, 79, 79, 80, 79, 80, 80, 79, 80, 79, 80]}. From this we can roughly evaluate the fairness of the overall schedule for each employee. 
    \item \textbf{Fairness: Avg Hours Worked (Per Shift) (STD):} Standard deviation of the average hours worked per shift. An example output looks like \texttt{[0.0, 0.8784226190040044, 0.7208809596176432, 0.4994114182774381]}. Since the domain of hours per shift tends to be 4 - 8 hours, a standard deviation roughly around 1 is expected. 
    \item \textbf{Fairness: Total Night Shifts Worked (Per Employee):} The total number of night shifts worked per employee. An example output looks like \texttt{[2, 2, 2, 2, 2, 2, 2, 2, 2, 2, 2, 2, 2, 2, 2, 2, 2]}. From this we can monitor whether or not our solutions overassign night shifts to any singular employee. 
    \item \textbf{Fairness: Total Off Shifts (Per Employee):} The total number of off shifts per employee. An example output looks like \texttt{[3, 2, 3, 2, 3, 3, 4, 4, 4, 3, 4, 3, 3, 4, 2, 3, 3]}. Same as above, we can monitor whether or not our solutions underassign off shifts to any singular employee.
    \item \textbf{Efficiency: AVG Distribution over shifts (per Employee):} The percent distribution of shifts worked, this moreso is a consequence of \texttt{min\_shifts} but it should reflect a healthy amount of \textt{off\_shifts} and \textt{night\_shifts} relative to other shifts. An example output looks like \texttt{[0.22268907563025211, 0.14285714285714285, 0.3067226890756303, 0.3277310924369748]}.
    \item \textbf{Efficiency: AVG Hours Worked (Per Shift):} The average number of hours worked per shift. An example output looks like \texttt{[0.0, 6.911764705882353, 7.501960784313725, 7.416666666666666]}.
    \item \textbf{Efficiency: AVG Extra Employees on Shift (Per Shift):}: The average number of extra employees on each shift. An example output looks like \texttt{['N/A', 0.6428571428571429, 0.07142857142857142, 2.7142857142857144]}.
\end{itemize}

Over all of the soltions that we solve our algorithm seems to score relatively high on fairness across employees
and alright on efficiency over shifts (The algorithm greatly prefers shift 3 due to the effects of our valueselector in our search heuristics). 
More specifically, employees tend to work roughly the same amount of night shifts and off shifts and approximately (with a std of 1 hour) the same amount of hours per shift. 
They also tend to work about 7 - 8 hours per shift indicating high efficiency per shift (but oftentimes shift 3 would be overfilled i.e. has the most slack). 
